\documentclass[a4paper,10pt,capitulo,azul,titlepage=false]{upeu-reg}

\usepackage[spanish]{babel}
\usepackage{geometry}
\usepackage{setspace}
\spacing{1}
\geometry{
 a4paper,
 total={170mm,257mm},
 left=27mm,
 right=25mm,
 top=25mm,
 bottom=20mm,
 }

\usepackage{fontspec}
\setmainfont{Arial}



\begin{document}

\tableofcontents

\newpage


\titulog{REGLAMENTO GENERAL UPEU}
\chapter{DE LA UNIVERSIDAD}
\section{DE LA IDENTIDAD, ORIGEN Y autonomía}
\article 
\descripcion{UPeU: Elementos de la comunidad. La Universidad Peruana Unión (UPeU) se 
define como una comunidad porque no reconoce, ni es constitutivo, ni connatural 
a su identidad los asociados, socios, accionistas o inversionistas. \\
Es una comunidad académica integrada por estudiantes, graduados, docentes y 
la participación de los representantes de su Promotora.\\
La UPeU en su naturaleza y carácter de comunidad, reconoce, de propósito 
afianza y evalúa, sus fines y funciones en razón de:}

\paragrafo En la numeración de las leyes se observarán además los siguientes criterios:
\paragrafo En la numeración de las leyes se observarán además los siguientes criterios:
\paragrafo En la numeración de las leyes se observarán además los siguientes criterios:

\section{La estructuración de las leyes}
\article Presuponer  que  el  ser  humano,  la  persona  humana,  es  el  centro  de  la 
atención, objeto de la mayor protección y consideración en cualquier faceta 
o escenario.

\paragrafo En la numeración de las leyes se observarán además los siguientes criterios:
\paragrafo En la numeración de las leyes se observarán además los siguientes criterios:
\paragrafo En la numeración de las leyes se observarán además los siguientes criterios:


\chapter{Definiciones preliminares}
\section{La estructuración de las leyes}
\article La ley se estructurará en tres partes básicas:
\paragrafo En la numeración de las leyes se observarán además los siguientes criterios:
\paragrafo En la numeración de las leyes se observarán además los siguientes criterios:
\paragrafo En la numeración de las leyes se observarán además los siguientes criterios:


\end{document}